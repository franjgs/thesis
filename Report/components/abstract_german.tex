% German abstract
% Included by main.tex

\clearemptydoublepage

\vspace*{2cm}
\begin{center}
{\Large \bf Zusammenfassung}
\end{center}
\vspace{1cm}

Nach Angaben der Weltgesundheitsorganisation sterben j\"ahrlich bis zu einer Million Menschen an Selbstmord. Dies bedeutet, dass jede 40 Sekunden ein Toter zu vermelden ist. Der Gro{\ss}teil aller Selbstmordopfer liegt im Alter zwischen 15 und 29 Jahren und wird somit zu der jungen Generation unserer Gesellschaft gez\"ahlt. Im Zuge des Erfolgs von sozialen Netzwerken wie Twitter, Facebook, Reddit und Wordpress hat sich in den letzten Jahren der Trend abgezeichnet, dass immer mehr Jugendliche dazu neigen ihre inneren Emotionen im Internet preiszugeben. Unabh\"angig davon, ob es sich um positive oder negative Gef\"uhle handelt. Um diese beiden Informationen miteinander in Kontext zu stellen, pr\"asentieren wir ein Kontroll- und \"Uberwachungssystem f\"ur Selbstmorde.\\

Das System verbindet \"uberwachte machine learning Techniken mit semi-online learning um Prognosen \"uber das generelle Stresslevel (in \%) von Twitter-Benutzern aufzustellen. Zudem werden bestimmte Beispiel-Tweets aufgezeigt, die dazu f\"uhren, dass eine Person anhand seiner Posts als depressiv diagnostiziert werden kann und somit folglich weitere Hilfe, wenn n\"otig, geleistet werden k\"onnte. Dieses System bedient sich au{\ss}erdem an Crowd Intelligence um eine fundierte Aussage dar\"uber zu treffen, welche Texte als depressiv gelten und welche nicht. Zus\"atzlich pr\"asentieren wir eine Einsch\"atzung zu Support Vector Machines und Ensemble-Lernmethoden im Bereich von Textklassifizierung und Emotionsanalyse. Das Ziel des finalen Systems ist, dass eine zeitnahe Intervention erm\"oglicht wird, sowie die Verbesserung des Gesundheitszustandes der Betroffenen.
