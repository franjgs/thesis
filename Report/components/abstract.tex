% Abstract for the TUM report document
% Included by main.tex

\clearemptydoublepage
\phantomsection
\addcontentsline{toc}{chapter}{Abstract}

\vspace*{2cm}
\begin{center}
{\Large \bf Abstract}
\end{center}
\vspace{1cm}

According to the World Health Organization, nearly one million people die from suicide every year. This means one death almost every 40 seconds. The maximum number of people committing suicide are between 15 and 29 years old, which comprises the young section of the society. The recent success of social networking websites like Twitter, Facebook, Reddit, and Wordpress has shown that more and more young people these days tend to express their inner feelings on the web, irrespective of whether those feelings are positive or negative. In order to bridge the disconnect between these two pieces of information, we present a surveillance and monitoring system of suicide.\\

The system blends supervised machine learning techniques and semi-online learning to make predictions about a general level of distress (in \%) amongst people posting on Twitter, and about particular instances of tweets which may lead one to believe that the author posting that content is depressed and may need further attention. The system also taps into crowd intelligence to learn about which pieces of text are depressed and which are not. In addition, we also present an evaluation of support vector machines and ensemble learning methods in the domains of text classification and sentiment analysis. The goal of the final system is to provide timely intervention and to promote better public health.
