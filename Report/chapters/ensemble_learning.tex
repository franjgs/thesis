\chapter{Ensemble Learning}
\label{chapter:Ensemble Learning}

Ensemble Learning is a class of supervised learning algorithms that use multiple models instead of a single one to obtain better predictions. A common problem with classifiers is that not all classifiers are suited for all kinds of problems. Ensembles combine multiple classifiers to construct a (hopefully) better classifier. The performance of an ensemble is not guaranteed to be better than the individual classifiers, but depending on the problem, an ensemble usually outperforms its constituent classifiers by a fair margin. Using ensemble learning requires performing more computation, but ensemble learning provides better results when the classifiers are not similar to each other. In this chapter, we explore the main ensemble learning techniques that we employed in our work.

\section{Bagging}
Bagging, also referred to as Bootstrap Aggregation is one of the most basic forms of ensemble methods. Given a training dataset $D$ of $n$ samples, each sample having $f$ different features, the aim of bagging is to combine a certain number of classifiers, each trained using a (mostly different) subset of the main data, such that every classifier learns about a different structure of the input data, and when the outputs from all such classifiers is combined, the final output is expected to be better than the individual predictions. Bagging is known to improve the performance when the classifier is slightly better than a random classifier. On the other hand, in cases where the constituent models are already strong, the performance is known to often degrade.

To train different classifiers on a different subset of the data, the training data is either split on the number of samples, or on the number of features. In the first case, we pick $n' (< n)$ samples (with replacement) and each time train a new classifier using the new dataset. Since the data points are picked with replacement, some of the data points may be common to some classifiers. In the second case, we train every new classifier using $m' (< m)$ features (with replacement). Whenever the ensemble is required to make a prediction on a new sample, a majority vote from all the constituent classifiers is taken, which becomes the final output of the ensemble. Bagging is known to reduce variance and avoid the overfitting problem.

\section{Boosting}
Boosting aims to build an ensemble by iteratively training weak classifiers on a distribution, and then finally putting them all together to build a strong classifier. In the training phase, the classifiers are trained one by one. Each training instance is assigned a numeric weight, which is the same for all samples in the beginning. After the first classifier has been trained, the weights for the points which were misclassified are increased, and the resulting input data (along with the modified weight values) is fed to the second classifier. This process continues until the last classifier has been trained. Each classifier has an incentive to focus more on classifying correctly those points which the previous classifier classified incorrectly.

At all times, each classifier's performance is maintained in a vector (usually named $\alpha$). Whenever a new point has to be classified, the individual predictions are obtained from the constituent classifiers, and a weighted sum (from the $\alpha$ values) is taken across all the classifiers to obtain the final prediction. One of the very popular Boosting techniques is AdaBoost. The main aim of Boosting is to build a strong classifier from a set of weak classifiers. To that effect, even classifiers only slightly better than random are considered useful, because in the final predictions, they will still contribute positively to the aggregate prediction by behaving like their inverses because of having negative coefficients in the final linear combination of classifiers.

\section{Stacking}
Stacking combines classifiers not by analyzing their performances on the training data, but by treating the individual predictions (on the training data) as a second level of training data, and then using this piece of data to build a final model that is ultimately used for making the final prediction. Different from the previous ensemble learning methods, stacking operates by feeding the initial training data to the set of base classifiers chosen. After these base classifiers have been well calibrated, the predictions on the training data are obtained from the base classifiers, forming a new dataset. This dataset is now fed into a second level classifier, which is responsible for making the final prediction. Often, the base classifiers (at the first level) are trained using different features of the input data to increase classifier diversity.

For making a prediction on a new point, predictions are made using all the base classifiers, forming an equivalent data point in which the number of features is the same as the number of base classifiers. This point is then run through the second level classifier to obtain the final prediction. In contrast to other ensemble learning methods, no voting or aggregation takes place in prediction. This class of methods has proven to be very successful, one recent example being that of the winning team in the Netflix prize competition.
